\chapter{Objects}
\label{ch:objects}

\section{Object Types}
\label{sec:objtypes}

\apsq provides a set of objects which can be used to transfer data between modules.
These objects can be sent with the messaging system as explained in Section~\ref{sec:objects_messages}.
A \texttt{typedef} is added to every object in order to provide an alternative name for the message which is directly indicating the carried object.

The list of currently supported objects comprises:


\nlparagraph{MCTrack}
The MCTrack objects reflects the state of a particle's trajectory when it was created and when it terminates.
Moreover, it allows to retrieve the hierarchy of secondary tracks.
This can be done via the parent-child relations the MCTrack objects store, allowing retrieval of the primary track for a given track.
Combining this information with MCParticles allows the Monte-Carlo trajectory to be fully reconstructed.
In addition to these relational information, the MCTrack stores information on the initial and final point of the trajectory (in \underline{global} coordinates), the energies (total as well as kinetic only) at those points, the creation process type, name, and the volume it took place in.
Furthermore, the particle's PDG id is stored.

\nlparagraph{MCParticle}
The Monte-Carlo truth information about the particle passage through the sensor.
A start and end point are stored in the object: for events involving a single MCParticle passing through the sensor, the start and end points correspond to the entry and exit points.
The exact handling of non-linear particle trajectories due to multiple scattering is up to module.
In addition, it provides a member function to retrieve the reference point at the sensor center plane in local coordinates for convenience.
The MCParticle also stores an identifier of the particle type, using the PDG particle codes~\cite{pdg}, as well as the time it has first been observed in the respective sensor.
The MCParticle additionally stores a parent MCParticle object, if available.
The lack of a parent doesn't guarantee that this MCParticle originates from a primary particle, but only means that no parent on the given detector exists.
Also, the MCParticle stores a reference to the MCTrack it is associated with.

MCParticles provide local and global coordinates in space for both the entry and the exit of the particle in the sensor volume, as well as local and global time information.
The global spatial coordinates are calculated with respect to the global reference frame defined in Section~\ref{sec:coordinate_systems}, the global time is counted from the beginning of the event.
Local spatial coordinates are determined by the respective detector, the local time measurement references the entry point of the \emph{first} MCParticle of the event into the detector.

\nlparagraph{DepositedCharge}
The set of charge carriers deposited by an ionizing particle crossing the active material of the sensor.
The object stores the \underline{local} position in the sensor together with the total number of deposited charges in elementary charge units.
In addition, the time (in \textit{ns} as the internal framework unit) of the deposition after the start of the event and the type of carrier (electron or hole) is stored.

\nlparagraph{PropagatedCharge}
The set of charge carriers propagated through the silicon sensor due to drift and/or diffusion processes.
The object should store the final \underline{local} position of the propagated charges.
This is either on the pixel implant (if the set of charge carriers are ready to be collected) or on any other position in the sensor if the set of charge carriers got trapped or was lost in another process.
Timing information giving the total time to arrive at the final location, from the start of the event, can also be stored.

\nlparagraph{PixelCharge}
The set of charge carriers collected at a single pixel.
The pixel indices are stored in both the $x$ and $y$ direction, starting from zero for the first pixel.
Only the total number of charges at the pixel is currently stored, the timing information of the individual charges can be retrieved from the related \parameter{PropagatedCharge} objects.

\nlparagraph{PixelHit}
The digitised pixel hits after processing in the detector's front-end electronics.
The object allows the storage of both the time and signal value.
The time can be stored in an arbitrary unit used to timestamp the hits.
The signal can hold different kinds of information depending on the type of the digitizer used.
Examples of the signal information is the 'true' information of a binary readout chip, the number of ADC counts or the ToT (time-over-threshold).

\section{Object History}
\label{sec:objhistory}

Objects may carry information about the objects which were used to create them.
For example, a \parameter{PropagatedCharge} could hold a link to the \parameter{DepositedCharge} object at which the propagation started.
All objects created during a single simulation event are accessible until the end of the event; more information on object persistency within the framework can be found in Chapter~\ref{ch:objects_persistency}.

Object history is implemented using the ROOT TRef class~\cite{roottref}, which acts as a special reference.
On construction, every object gets a unique identifier assigned, that can be stored in other linked objects.
This identifier can be used to retrieve the history, even after the objects are written out to ROOT TTrees ~\cite{roottree}.
TRef objects are however not automatically fetched and can only be retrieved if their linked objects are available in memory, which has to be ensured explicitly.
Outside the framework this means that the relevant tree containing the linked objects should be retrieved and loaded at the same entry as the object that request the history.
Whenever the related object is not in memory (either because it is not available or not fetched) a \parameter{MissingReferenceException} will be thrown.

A MCTrack which originated from another MCTrack is linked via a reference to this track, this way the track hierarchy can be obtained.
Every MCParticle is linked to the MCTrack it is associated with.
A MCParticle can furthermore be linked to another MCParticle on the same detector.
This will be the case if there are MCParticles from a primary (parent) and secondary (child) track on one detector.
The corresponding child MCParticles will then carry a reference to the parent MCParticle.
